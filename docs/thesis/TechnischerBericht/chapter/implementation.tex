\chapter{Implementation}
\thispagestyle{main} % Needed for Footer and Header on Chapterpage
This chapter focuses on explaining why we did what. We describe the trade-offs and decisions we've taken while building the VM, and describe them in a manner that follow up works get a better understanding of why some parts of the VM are implemented in a certain way so that they either are more secure when changing something or to help them reach the same conclusion faster.

\section{Stack}

% TODO insert crc card for stack

\subsection{Maximum stack size}
Facing the concern of excess memory usage of the contract on the miner, we decided to limit the stack size to 1MB which seems to be well above what the contracts will need. We neglected using the gas amount for max storage determination because it would be just a soft limit

\subsection{Maximum variable size}
Since it is possible to easily craft denial of service attacks for the blockchain if there is no limit on the variable size we decided to set the variable size to a specific fixed amount instead of arbitrary precision.

\subsection{Data structure}
As underlying data structure of "Stack", keeping in mind to keep the code of the vm short and simple, we decided to use an array of big.Int. It was important that the datatype used for the underlying datatype was of arbitrary length because splitting up an element into multiple bytes so that it can be saved when using an array of bytes as underlying data structure would have been a lot more complex and more error-prone. Arbitrary length is implemented for example an array of jagged arrays of bytes. On the vm it was very important that the type used in the vm provided mathematical methods and also that it could store and work with values of up to 256 bit size. Only big.Int fulfilled both requirements and therefore it was choosen for the vm. Thanks to big.Int being of arbitrary length we could then use an array of big.Int as underlying data structure which also simplifies our code since we don't have to cast the values retrieved from the stack before working with them.

We neglected working with pointers because even though there are more elements created on the heap of the physical machine, it shouldn't make a difference considering the vast availability of resources on modern computers and our rather small contracts. We neglected using a simple array of bytes as the whole data structure where different datatypes are read using a different amount of bytes, as it is common when having no abstraction layer or using a jagged array of byte arrays because the could becomes a lot more readable in the vm and less conversions from big.Int to byte array are necessary. We accept the dependency on the datatype big.Int which is necessary as it implements all mathematical operations which are very important and because it is of arbitrary precision and the size of other datatypes would not have been sufficient especially for cryptographic purposes.

\section{Virtual machine}

% TODO insert crc card for stack
\subsection{Error handling}
The VM does what it's told to do from the compiled bytecode. If there are any errors while executing an opCode an error value is provided. The vm pushes the error message on the Stack and halts.

Since the VM operates on copies the values will also not be changed by a faulty contract.

\subsection{opCodes}
% TODO add opcodes table