\begin{abstract}
	\thispagestyle{main}

%Der Abstract richtet sich an den Spezialisten auf dem entsprechenden Gebiet und beschreibt daher in erster Linie die (neuen, eigenen) Ergebnisse und Resultate der Arbeit. Es umfasst nie mehr als eine Seite, typisch sogar nur etwa 200 Worte (etwa 20 Zeilen). Es sind keine Bilder zu verwenden.

%Welches Problem wird mit der Untersuchung gelöst?
Die Arbeit befasst sich mit der Integration von Smart Contracts in die BAZO Blockchain.
Die BAZO Blockchain begann als Forschungsprojekt der Universität Zürich in Zusammenarbeit mit einem Finanzdienstleister und wird nun unabhängig von Prof. Dr. Thomas Bocek weitergeführt.
%Warum ist die Untersuchung wichtig? 
Smart Contracts bieten chancen in der Automatisierung indem auf gewisse preconditions automatisch Transaktionen ausgelöst werden statt manuell. Speziell interessant ist dies, wenn viele kleine Transaktionen gemacht werden müssen.
Weitere Chancen gibt es im Rahmen der Dezentralisierung. Verträge können in der Blockchain als Smart Contract gespeichert werden, die Ausführung ist durch die Eigenschaften der Blockchain garantiert.
%Wie wird das Problem gelöst?
Gelöst wurde die Integration von Smart Contracts über die Implementation und Integration eines Interpreters bzw. einer Virtuellen Maschine für benötigte byte codes die später in einer Transaktion gespeichert werden.
%Welches sind die wichtigsten Ergebnisse? 
Als resultat können Transaktionen erstellt werden die einen Smart Contract enthalten und Transaktionen, die die Ausführung des Codes eines Smart Contracts bewirken, welcher dann von der Virtuellen Maschiene ausgeführt wird und ein entsprechendes Resultat im State des Miners speichert. 
%Wie sind sie zu bewerten?
Die BAZO Blockchain ist nach wie vor ein Forschungsprojekt, dies bedeutet auch, dass es viel einarbeitungszeit benötigt entsprechende Funktionalitäten zu nutzen
%Was bedeutet das für die künftige Arbeit in diesem Feld?	
Weiterführende Arbeiten könnten dies vereinfachen und eine eigene high level Sprache zur Erstellung von Smart Contracts implementieren.
	
English Abstract....
This thesis is concerned with the integration of smart contracts into the Bazo Blockchain with the goal of creating a platform for  decentralized applications. The Bazo Blockchain project started as a research project at the University of Zurich in cooperation with a financial service provider and is now continued independently by Prof. Dr. Thomas Bocek. Smart contracts are programs that are deployed and called by executing a transaction on a blockchain. Smart contracts offer opportunities in automation by automatically triggering transactions on certain events and bring along advantages provided by blockchains such as immutability, public visibility of transactions and decentralization. As blockchains are trust-less protocols, it is guaranteed that smart contracts are executed as intended. The integration of smart contracts was solved by implementing a virtual machine that executes the byte code instructions carried by a transaction. Furthermore the virtual machine was embedded within the mining application which required to alter the blockchain protocol and mining application itself. As a result, contract deployment transactions and transactions that call functions of smart contracts can be executed. Calling a smart contract function leads to the execution of the contract in the virtual machine and persisting the result as updated state variable. The Bazo Blockchain is still a research project, this means that it requires a lot of training to use the corresponding functionalities. Follow-up theses could simplify the development of smart contracts for the Bazo Blockchain by creating a compiler that compiles contracts written in a high-level programming language to Bazo Virtual Machine byte code instructions.

\end{abstract}
