\chapter{Installation Guidelines}
\section{Miner Application}
\subsection{Prerequisites}
The programming language Go (developed and tested with version >= 1.9) must be installed, the properties \$GOROOT and \$GOPATH must be set. For more information, please check out the official documentation.

Before the bazo-miner can be started, two public-private key-pairs are required. The key-pairs can be generated with the bazo-keypairgen application. Run the following instructions in your terminal.

\begin{enumerate}
	\item Download the bazo-keypairgen application.
	\begin{minted}[
		frame=lines,
		framesep=2mm,
		autogobble,
		baselinestretch=1.2,
		fontsize=\footnotesize,
		]
		{bash}
		go get github.com/bazo-blockchain/bazo-keypairgen
  		\end{minted}
	\item Build the application.
	\begin{minted}[
		frame=lines,
		framesep=2mm,
		baselinestretch=1.2,
		fontsize=\footnotesize,
		autogobble
		]
		{bash}
		$GOPATH/src/github.com/bazo-blockchain/bazo-keypairgen
		go build
  		\end{minted}
  	\item Run the application to generate the \textit{validator} public-private keypair. The validator is the keyfile\'s name containing the validator\'s public key.
  	\begin{minted}[
		frame=lines,
		framesep=2mm,
		baselinestretch=1.2,
		fontsize=\footnotesize,
		autogobble
		]
		{bash}
		./bazo-keypairgen validator.txt
  		\end{minted}
  	\item Run the application to generate the \textit{multisig} public-private keypair. The multisig is the keyfile\'s name containing the multi-signature server\'s public key.
  	\begin{minted}[
		frame=lines,
		framesep=2mm,
		baselinestretch=1.2,
		fontsize=\footnotesize,
		autogobble
		]
		{bash}
		./bazo-keypairgen multisig.txt
  		\end{minted}
\end{enumerate}

\subsection{Getting Started}
\begin{enumerate}
	\item Download the bazo-miner application.
	\begin{minted}[
		frame=lines,
		framesep=2mm,
		baselinestretch=1.2,
		fontsize=\footnotesize,
		autogobble
		]
		{bash}
		go get github.com/bazo-blockchain/bazo-miner
  		\end{minted}
	\item Copy both previously generated files \textit{validator.txt} and \textit{multisig.txt} into the root folder of the bazo-miner folder.
	\begin{minted}[
		frame=lines,
		framesep=2mm,
		baselinestretch=1.2,
		fontsize=\footnotesize,
		autogobble
		]
		{bash}
$GOPATH/src/github.com/bazo-blockchain/bazo-keypairgen
cp validator.txt $GOPATH/src/github.com/bazo-blockchain/bazo-miner/validator.txt 
cp multisig.txt $GOPATH/src/github.com/bazo-blockchain/bazo-miner/multisig.txt 
  		\end{minted}
  	\item Open the storage configuration file \textit{storage/configs.go} in an editor of your choice.
  	\begin{minted}[
		frame=lines,
		framesep=2mm,
		baselinestretch=1.2,
		autogobble,
		fontsize=\footnotesize,
		]
		{bash}
$GOPATH/src/github.com/bazo-blockchain/bazo-keypairgen
cp validator.txt $GOPATH/src/github.com/bazo-blockchain/bazo-miner/validator.txt 
cp multisig.txt $GOPATH/src/github.com/bazo-blockchain/bazo-miner/multisig.txt 
  		\end{minted}
  		Replace the value of \textit{INITROOTPUBKEY1} with the first line of \textit{validator.txt}. Replace the value of \textit{INITROOTPUBKEY2} with the second line of \textit{validator.txt}.
  	\item Build the application.
  	\begin{minted}[
		frame=lines,
		framesep=2mm,
		baselinestretch=1.2,
		fontsize=\footnotesize,
		autogobble
		]
		{bash}
		$GOPATH/src/github.com/bazo-blockchain/bazo-miner
		go build
  		\end{minted}
  	\item Run the application.
  	\begin{minted}[
		frame=lines,
		framesep=2mm,
		baselinestretch=1.2,
		fontsize=\footnotesize,
		autogobble
		]
		{bash}
		./bazo-miner "database_file.db" ":8000" "validator.txt" "seedfile.txt" "multisig.txt"
  		\end{minted}
  		The ipport number must be prefixed with ":". If the miner is intended to run locally, the localhost ip address has to be passed with the ipport. Otherwise the miner tries to connect to the network. Note that "database\_file.db" and "seedfile.txt" are created if they do not exist.
\end{enumerate}

\section{Parser Application}
\subsection{Prerequisites}
The programming language Go (developed and tested with version >= 1.9) must be installed, the properties \$GOROOT and \$GOPATH must be set. For more information, please check out the official Go documentation.
\subsection{Getting Started}
\begin{enumerate}
	\item Download the bazo-parser application
	\begin{minted}[
		frame=lines,
		framesep=2mm,
		baselinestretch=1.2,
		fontsize=\footnotesize,
		autogobble
		]
		{bash}
		go get github.com/bazo-blockchain/bazo-parser
  		\end{minted}
	\item Build the application.
	\begin{minted}[
		frame=lines,
		framesep=2mm,
		baselinestretch=1.2,
		fontsize=\footnotesize,
		autogobble
		]
		{bash}
		$GOPATH/src/github.com/bazo-blockchain/bazo-parser
		go build
  \end{minted}
  \item Run the application
  \begin{minted}[
		frame=lines,
		framesep=2mm,
		baselinestretch=1.2,
		fontsize=\footnotesize,
		autogobble
		]
		{bash}
		./bazo-parser
  \end{minted}
  \item	Define the path to the smart contract. After hitting enter, the parser prints the compiled byte code instructions, ready to copy to the virtual machine or the miner.
  \begin{minted}[
		frame=lines,
		framesep=2mm,
		baselinestretch=1.2,
		fontsize=\footnotesize,
		autogobble
		]
		{bash}
		Define the path to your contract
		./contracts/addNums.sc
  \end{minted}
\end{enumerate}