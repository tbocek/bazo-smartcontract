\chapter{Protocols}
\section{KW22:}
\subsection{Agenda}
\begin{itemize}
\item
  Are the requirements of the financial service provider full filled or
  is there still stuff to do?
\item
  Review system context diagram (what to do with the client?)
\item
  IFS or INS logo? and can it be used?
\item
  Checkout future work
\item
  Abstract
\item
  Problem definition
\item
  Poster
\item
  We'd like to take the testing over the web api out of scope in order
  to refactor and write on the thesis.
\item
  Which formats of the thesis are required? Print and pdf or only pdf?
\item
  Show tokenization contract.
\item
  marginalia
\item
  Can we send you chapters to read trough, not for error checking but
  formulation, style etc.
\item
  Merging Branches
\end{itemize}

\subsection{Notes}
\begin{itemize}
\item
  The requirements of the financial service provider are independant
\item
  reviewed diagram
\item
  Logo is IFS
\item
  Future work is ok
\item
  The English abstract is important, we'll send it later to read
\item
  Email for problem definition (Aufgabenstellung) is sent
\item
  Taking the testing of the web api out of scope is ok
\item
  Just a pdf would be good.
\item
  Marginalia are ok
\item
  Marc is still working for the financial service provider to finish the
  work so this should be ok to contact him over slack.
\end{itemize}

\section{KW21:}
\subsection{Agenda}
\begin{itemize}
\item
  Show integration tests
\item
  Integration of miner/client
\item
  Parser
\item
  Thesis
\end{itemize}

\subsection{Notes}
\begin{itemize}
\item
  Combining the client and the miner is more complicated than thought
  since both are cli tools. Thomas has a deeper look on how to combine
  them
\item
  Parser moved to own repo, adding a main file and turning it into a cli
  tool makes sense.
\item
  Thesis:
  \begin{itemize}
  \item
    Add gas price to opCode table
  \item
    Self-contained is more important than quantity of text
  \item
    Having a functionable product is most important
  \item
    Use package for code not image
  \item
    Suggest date for final presentation (27.6 is when the other group
    presents) → Tell Thomas if that's a suitable date for us.
  \end{itemize}
\end{itemize}

\section{KW20:}
\subsection{Agenda}
\begin{itemize}
\item
  What do we get Back from Transactions
  \begin{itemize}
  \item
    Errors
  \item
    When Getting parameters like balance etc.
  \item
    How to return something? *~Adjustments made to the Miner
  \item
    Encode/Decode, adapt changes to all encode/decode functions
    (eventually with a performance loss)?
  \end{itemize}
\item
  Adjustments of the Client
\item
  New Repository for Parser
\item
  Thesis Background and related work

  \begin{itemize}
  \item
    What to take into Design, how to structure design
  \end{itemize}
\end{itemize}

\subsection{Notes}
\begin{itemize}
\item
  Future work

  \begin{itemize}
  \item
    replace GOB encoding by platform independant encoding and change
    Encode/Decode function of every component.
  \item
    decouple fee from currency, so price of gas per coin can be set, in
    case the currency gains a lot in value
  \end{itemize}
\item
  Implement transaction recipe to check afterwords if transaction was
  successful (edit: 13.06. is already implemented)
\item
  Testing: test if state can be altered successfully, make second
  transaction to change the state again.
\item
  Merge client and miner, ask other organisation members beforehand
\item
  Use gob only for our components, since it is platform dependant
\item
  Make new repo for parser
\item
  Thesis:

  \begin{itemize}
  \item
    No summary needed, combine future work and conclusion
  \item
    Benchmarking direclty with go implementation, Ethereum and Neo
  \end{itemize}
\end{itemize}

\section{KW19}
\textit{Intermediate Presentation}

\section{KW18: spontaneous Meeting 03,05,2018}
\subsection{Agenda}
\begin{itemize}
\item
  Vm integration proposal

  \begin{itemize}
  \item
    Extend accounttx.go and fundstx.go with additional fiels
  \item
    Adjustment of encode and decode methods
  \item
    Executing smart contracts (VM.exec()) in the addFundsTx method of
    miner/block.go
  \end{itemize}
\end{itemize}

\subsection{Notes}
\begin{itemize}
\item
  Integration procedure should be okay
\item
  Maybe implement a feature which executes the vm on the client
  firsthand and if that result executed on the copies is ok then
  actually create the transaction.
\item
  Making contracts able to call further contracts.
\end{itemize}

\section{KW17}
\textit{No meeting}

\section{KW16: Meeting 19,04,2018}
\subsection{Agenda}
\begin{itemize}
\item
  Datastructures

  \begin{itemize}
  \item
    Functions
  \item
    Internals
  \end{itemize}
\item
  Zwischenpräsentation

  \begin{itemize}
  \item
    Which questions does Thomas want to have answered
  \end{itemize}
\item
  Syntax of simplified byte code
\item
  Parser
\item
  Miner integration
\item
  ABI
\item
  Small smart contracts
\end{itemize}

\subsection{Notes}

\begin{itemize}
\item
  Datastructures\\
  \textbf{Array:}

  \begin{itemize}
  \item
    Based on go array struct
  \item
    Loop for index access
  \item
    Bytearray is converted to BigInt and appended with length of
    bytearray
  \item
    Array entry has size, data
  \item
    uint16 size
  \end{itemize}

  \textbf{Map:}

  \begin{itemize}
  \item
    Analogous to array
  \item
    Entry has key length, key, element size and element
  \end{itemize}

  \textbf{Struct:}

  \begin{itemize}
  \item
    Array can be used to represent structs
  \end{itemize}
\item
  ABI: switch case would be more efficient but if else is much easier to
  implement and analogous to Ethereum.
\item
  Zwischenpräsentation:~Show how a simple contract (addition of two
  numbers) execution works in detail from begin to end.
\item
  Integration meeting at Bahnhofstrasse 3
\item
  Implement smart contract variables (data) to be written back atomar
  with the HALT opCode to be able to reset changes.
\end{itemize}

\subsection{Todos}

\begin{itemize}
\item
  Contracts to define:

  \begin{itemize}
  \item
    Concurrency Contracts
  \item
    Benchmark Tests
  \end{itemize}
\item
  Refactoring
\item
  OpCodes for context data (get the sender address on top of stack,...)
\item
  Extend CALLEXT~opCode to send amount of bazo coins
\item
  Gas calculation should take size of elements on stack into account
\end{itemize}

\section{KW15: Meeting 12,04,2018}
\subsection{Agenda}

\begin{itemize}
\item
  Changes made to the code

  \begin{itemize}
  \item
    Datastructures.
  \item
    Fuzzing
  \end{itemize}
\item
  Discuss idea how to implement struct
\item
  Outline ok?
\end{itemize}

\subsection{Notes}

\begin{itemize}
\item
  Map has variable key and value length

  \begin{itemize}
  \item
    Datastructures are not solved with serialization
  \item
    Uses byte data structure
  \item
    Not solved with adding memory
  \end{itemize}
\item
  Size of stack elements that are manipulated has to be taken into
  account when calculating gas cost
\item
  Outline okay
\end{itemize}

\subsection{Todos}

\begin{itemize}
\item
  Finish implementation of data structures
\item
  Benchmark with large numbers and compare to Ethereum, NEO

  \begin{itemize}
  \item
    Modular exponentiation
  \end{itemize}
\item
  Take size of elements into account when calculating gas cost
\item
  Add evaluation chapter to thesis
\item
  Make proposition for Zwischenpräsentation
\end{itemize}

\section{KW14: Meeting 05,04,2018}

\subsection{Agenda}

\begin{itemize}
\item
  Changes made to the code

  \begin{itemize}
  \item
    Show negate operation.
  \end{itemize}
\item
  Data structure to store contract variables - discussion
\item
  Thesis
\item
  Contact data of mchetelat
\item
  What is meant by integratable into the miner
\item
  Fuzz-Testing
\item
  How to handle panic if PUSH 2 but only one byte parameter?
\item
  Storage: Key/Value, Hash as Key or enumerate? Or simply Slice of
  big.Ints? Or even BoltDB?
\end{itemize}

\subsection{Notes}
\begin{itemize}
\item
  The miner shouldn't crash if arbitrary operations are executed. Fuzz
  Testing
\item
  Arrange meetings and conversations with mchetelat
\item
  When the current tasks are finished the miner can be integrated.
\item
  Thesis ok for now
\item
  Use slice of Big.Ints for storage
\end{itemize}

\subsection{Todos}
This weeks tasks are under projects in the github repository

\section{KW13: Meeting 26,03,2018}

\subsection{Agenda}

\begin{itemize}
\item
  Review changes made since last meeting

  \begin{itemize}
  \item
    General vm structure

    \begin{itemize}
    \item
      byteArray, int64 based calculation
    \item
      callStack
    \item
      Memory
    \item
      trace function
    \item
      opCode Bitshift R/L
    \end{itemize}
  \item
    Execution Context
  \item
    Gas implementation

    \begin{itemize}
    \item
      Subtract every loop and subtract more on SSTORE~and SLOAD?
    \end{itemize}
  \item
    Errorhandling should not interrupt VM

    \begin{itemize}
    \item
      Handle every error in VM~and jump to HALT/additional THROW~opCode?
    \end{itemize}
  \end{itemize}
\item
  How to insert data from account into context?
\item
  Automatically compile ABI?
\item
  Write parser to match address to labels? (CALL addition 2, instead of
  CALL~13 (address where function starts) 2)
\item
  Load manually in function or automatically according to passed nargs?
\item
  Does passing context make sense?

  \begin{itemize}
  \item
    Or better Context as VM Field?
  \end{itemize}
\item
  Stack references instead of instance?
\item
  Better way to structure packages?
\item
  Roll~Function understanding
\item
  Zwischenpräsentation?
\end{itemize}

\subsection{Notes}

\begin{itemize}
\item
  Change uint64 based operations to Big.Int would be better.
\item
  Think about why callStack is necessary and what advantages it brings
  compared with only evaluation stack based implementation.

  \begin{itemize}
  \item
    Call Stack enables recursion
  \item
    Return address could also be pushed on the stack
  \end{itemize}
\item
  Memory (MLOAD and MSTORE) is unnecessary. Everything should be done on
  the evaluation stack and with opCode DUP, ROLL
\item
  Because all Storage operations have to be reversed if not enough gas
  is available, writing to storage (SSTORE) happens at the end of
  contract execution and only if there is enough gas left.
\item
  Every operation costs the same and is generally cheap. SLOAD is more
  expensive (about factor 100), SSTORE is a lot more expensive (about
  factor 10'000), inspired by Ethereum Yellow Paper
\item
  Error-/Exceptionhandling:~If an error occurs the VM~must catch it,
  halt the transaction with error code and be able to accept the next
  transaction without restarting. \texttt{Exec()} returns bool to show
  if transaction was successful.

  \begin{itemize}
  \item
    Check err object in Exec Function, if Error, go to Error~Halt
  \end{itemize}
\item
  Context should be a VM~struct field instead of being passed to Exec
  function
\item
  Stack implementation should be changed from instance based to
  reference based to achieve higher performance. It is important that
  before accepting a new transaction the VM~must be reseted.
\item
  Ignore how to access account data for now and work with mock and dummy data
\end{itemize}

\subsection{Todos}
\begin{table}[H]
\centering
\begin{tabular} {L{1cm} L{3cm} L{6cm} L{1.5cm} R{1.5cm} R{1cm}}
\toprule
\textbf{\#} & \textbf{Notes}        & \textbf{Action}                                                                           & \textbf{Assigned to} & \textbf{Deadline} \\ \midrule
1           & Gas Counter           & Implement gas counter to work as specified in notes                                       & Marco                & KW14              \\
2           & Error handling        & Implement error handling as specified in notes                                            & Ennio                & KW14              \\
3           & opCodes               & Implement ROLL, DUP and opCode to call external functions                                 & Ennio                & KW14              \\
4           & Reference based Stack & Move from instance based stack to reference based                                         & Marco                & KW14              \\
5           & uint64 to Big.Int     & Move from uint64 based int operations to Big.Int                                          & Marco                & KW14              \\
6           & Context as Field      & Context as Field instead of parameter of Exec                                             & Marco                & KW14              \\
7           & Memory, callStack     & Replace memory with dup, roll opcodes, think if callStack has any advantages and document & Ennio                & KW14              \\ \bottomrule
\end{tabular}
\end{table}

\section{KW12}
\textit{No meeting}

\section{KW11: 15.03.2017}
\subsection{Agenda}

\begin{itemize}
\item
  General structure
\item
  How to access variables from outside contract?~(i.e. msg.sender, Read
  from DB into stack, modify Peek to pass index parameter?)
\item
  Size of Data Field in Smart Contract (extends Account)
\item
  What does account size mean?
\item
  Do the methods in the contract have to be polymorphic?
\item
  How are accounts stored? They only seem to be instantiated after a
  transation into the memory and a map holds a pointer to the account
  object but they are never persisted.
\item
  Is it possible to call the function with an index to the byte code at
  which the function starts?
\item
  How should we implement the ABI? Map with function name and
  instruction index.
\end{itemize}

\subsection{Notes}

\begin{itemize}
\item
  General structure

  \begin{itemize}
  \item
    Input data is function hash and parameters
  \item
    Block header field is needed for block number/hash
  \end{itemize}
\item
  State changes are transmitted from memory to storage at the end of the
  contract execution
\item
  Smart contract functions should be callable from other smart
  contracts: opCode CALL, implementation not high priority
\item
  Using hashes of function signature to identify functions would allow
  polymorphism and method overriding.~Another approach is enumerating
  the functions or the function signature as string.
\item
  Everything is done on the memory, writing to storage happens manually
  at the end of the smart contract execution.
\item
  Datatypes:

  \begin{itemize}
  \item
    either int265 (simple to implement)~or different int sizes (32, 64,
    256Bit), which is harder to implement but leads to less memory
    utilisation.
  \item
    Float should not be implemented, because calculating leads to a loss
    in precision.
  \item
    Datatypes should not be stored on stack, Pop
  \end{itemize}
\item
  OpCode SLOAD, SSTORE, MLOAD, MSTORE~should use enumeration as key
  instead of string.
\item
  Reduce Stack to only one Pop function which returns
  \texttt{{[}{]}byte}, conversion from \texttt{{[}{]}byte} to int/string
  is done in the \texttt{Exec()} function.
\item
  Account Transaction size must be changed from static size to dynamic
  size.
\item
  Store SHA3 hash as \texttt{{[}{]}byte} instead of string
\item
  Rethink how to convert bytecode to {[}{]}instruction~(Instruction
  interpreter)
\item
  Length of argument should be included in opCodes (PUSHI/PUSHS)
\item
  Gas usage for storage operations higher, other operations cost all the
  same
\item
  VM exception handling: If exception occurs, the VM~halts
  (opCode:~HALT) transaction and the next is executed.
\end{itemize}

\subsection{Todos}
\begin{table}[H]
\centering
\begin{tabular}{L{1cm} L{3cm} L{6cm} L{1.5cm} R{1.5cm} R{1cm}}
\toprule
\# & Notes                                               & Action                                                                                                                    & Assigned to & Deadline \\ \midrule
1  & Instruction interpreter                             & Build function to convert from Bytecode to {[}{]}instructions and vice versa                                              & Ennio       & KW12     \\
2  & opCodes                                             & Define opCodes for memory and storage, think about RET (with return value, without return value to jump out of function?) & Ennio       & KW12     \\
3  & ABI                                                 & Implement ABI directly as switch case at the beginning of smart contract                                                  & Marco       & KW12     \\
4  & Shift Left/Right and Roll (yenga function) on stack & Implement functions on Stack and opCodes                                                                                  & Marco       & KW12     \\
5  & Define limits                                       & Define limits for PC and stack size                                                                                       & Both        & KW13     \\ \bottomrule
\end{tabular}
\end{table}

\section{KW10: Meeting 08.03.2018}
\subsection{Agenda}
\begin{itemize}
\item
  Present VM, critics to VM
\item
  Specify Datatypes more precise? (i.e. uint32 instead of int?)
\item
  How to proceed? Finish VM or make changes to blockchain first and test
  very basic smart contract execution?
\item
  How to call smart contract function?
\item
  How to access variables from outside contract? (i.e. msg.sender, Read
  from DB into stack, modify Peek to pass index parameter?)
\item
  Name of thesis? (Smart contracts on bazo blockchain)
\item
  Request bazo-blockchain travis access, to add CI
\end{itemize}

\subsection{Decisions}
\begin{itemize}
\item
  Present VM, critics to VM
\item
  Specify Datatypes more precise?

  \begin{itemize}
  \item
    Use jagged byte array.
  \end{itemize}
\item
  How to proceed?

  \begin{itemize}
  \item
    Specify simple smart contract in solidity and adapt VM to execute
    equivalent version of the contract.
  \item
    Specify new opCodes: Hash algorithm (same as miner), opcode to
    handle state
  \item
    Include state and improve VM datatypes
  \item
    State itself is saved in account (Use JSON for first version)
  \end{itemize}
\item
  How to call smart contract function?

  \begin{itemize}
  \item
    Solidity uses function hashes -\textgreater{} We could implement it
    using Call to index of the start program index of the function.
  \end{itemize}
\item
  Name of thesis?

  \begin{itemize}
  \item
    Smart Contracts for the Bazo Blockchain
  \end{itemize}
\end{itemize}

\section{KW09: Meeting 01.03.2018}

\subsection{Agenda}

\begin{itemize}
\item
  How do you want to use Smart Contracts?
\item
  How should the Gas be calculated?
\item
  Explanation or Documentation of the miner on the BAZO organisation
\item
  How to get from High Level Code to "Opt Codes" (Interpreter?) check if
  understood correctly.
\item
  Which changes have to be made at the Blockchain (another transaction
  type in client)
\item
  Use Neo VM or directly create and implement own VM.
\item
  Factors for Language, what makes a good language for you?
\item
  Project plan
\item
  Risks
\end{itemize}

\subsection{Decisions}

\begin{itemize}
\item
  Discussions rather over instant messaging, important things which have
  to be acted on over email.
\item
  Implement two kinds of accounts, similar to Ethereum. Externally Owned
  accounts and Smart Contract accounts. Smart contract accounts are
  owned and triggered by externally owned accounts.
\item
  Focus on smart contracts using on-chain data only.
\item
  Using 32 bit as minimal makes sense (OpCodes).
\item
  TxFunds probably must be extended with an additional data field which
  contains OpCodes.
\item
  Factors languages for smart contracts should meet:

  \begin{itemize}
  \item
    Strong typed
  \item
    Not too permissive
  \item
    Possibility:~write own language
  \end{itemize}
\item
  Generally to keep in mind during this project:~Keep it Simple!
\item
  Idea to consider:~smart contract pays the fee instead of the executer
\item
  How to proceed: Start defining OpCodes and writing the interpreter in
  Go (MVP).
\end{itemize}

\subsection{Todos}
\begin{table}[H]
\centering
\begin{tabular}{L{1cm} L{3cm} L{6cm} L{1.5cm} R{1.5cm} R{1cm}}
\toprule
\# & Notes               & Action                                                                             & Assigned to & Deadline \\ \midrule
1  & Github organization & Thomas invites us to the Bazo Blockchain Github organization.                      & Thomas      & KW10     \\
2  & Miner parameters    & Thomas inquires why the parameter are needed                                       & Thomas      & KW10     \\
3  & Logo                & Ennio sends Thomas the logo as svg                                                 & Ennio       & KW10     \\
4  & Time management     & Last two week team works full-time on project, therefore move back code freeze MS. & Team        & KW10     \\ \bottomrule
\end{tabular}
\end{table}

\section{KW08: Kickoff Meeting 22.02.2018}
\subsection{Agenda}

\begin{itemize}
\item
  Definition of Scope
\item
  Project management

  \begin{itemize}
  \item
    VCS
  \item
    Meetings
  \item
    Protocols
  \item
    Issue tracking
  \end{itemize}
\item
  Documents to hand in
\item
  Preparation

  \begin{itemize}
  \item
    Blockchain technology (generic and bazo)
  \item
    Golang
  \item
    EVM, LLVM
  \end{itemize}
\end{itemize}

\subsection{Decisions}
\begin{itemize}
\item
  Definition of Scope\\
  Team needs to read into the matter first, in order to specify the
  scope of this project. Goal is to add the capability of executing
  smart contracts to the bazo blockchain, either by using an existing VM
  or by building an interpreter (LLVM).
\item
  Project management\\
  Is mainly decided by the dev-team. As VCS we use github, protocols are
  written in the wiki, issue tracking is done with github wiki. Time
  tracking is done in an external excel file.
\item
  Documents to hand in

  \begin{itemize}
  \item
    Project plan by next week, needs to contain deadlines, infos about
    weekly meetings, scope of the project.
  \item
    Outline by the middle of the project, shows the structure of the
    document
  \item
    Draft by 2/3 of the project
  \end{itemize}
\item
  Licenses

  \begin{itemize}
  \item
    License of the source code is Apache 2.0
  \item
    License of the thesis is Creative Commons
  \end{itemize}
\item
  Testing

  \begin{itemize}
  \item
    Test-driven design should be used
  \end{itemize}
\end{itemize}

\subsection{Todos}
\begin{table}[H]
\centering
\begin{tabular}{L{1cm} L{3cm} L{6cm} L{1.5cm} L{1.5cm}}
\toprule
\# & Notes               & Action                                            & Assigned to & Deadline \\ \midrule
1 & Project plan & Write the project plan as described above.                & Both & KW09    \\
2 & Learn        & Get familiar with smart contracts, ethereum, LLVM, golang & Both & KW09/10 \\ \bottomrule
\end{tabular}
\end{table}