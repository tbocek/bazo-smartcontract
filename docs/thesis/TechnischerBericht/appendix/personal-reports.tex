\chapter{Personal Reports}
\section{Ennio Meier}
I've never worked with my partner before this project. After a short talk we decided to do the thesis together. After the presentation of the topics and a subsequent discussion with Thomas it was soon clear that we wanted to do this thesis. The cooperation worked very well. We were both very committed and dealt with the subject thoroughly. Of course, there were also disagreements, but we were always able to reach a compromise. I think that a willingness to compromise and the ability to take criticism are also important learning objectives of a joint Bachelor's thesis. 

This thesis was very contrary to my student research project. In this thesis, the focus was much more on implementation. We also worked truly agile and not according to the waterfall principle, which worked very well for me. I was able to learn a lot in this thesis, especially in the area of blockchain, about how computers work on a deeper level and in the area of testing, in which many practices known from theory were applied. The workload was very balanced and the tasks my partner has carried out have always been of high quality. I am proud that we have completed this thesis and I'm grateful to have done this work together. I believe that we have done a good job, which helps the Bazo Blockchain to grow.

Thomas has supported us from the very beginning. Thomas always gave us meaningful and well thought-out feedback, which shows that he is an expert in this field. He was often available for spontaneous meetings and even took us to Trust Square, a co-working space for blockchain companies. 

\section{Marco Steiner}
At the start of the work I did not really have a lot experience with blockchain or virtual machines. Therefore, I was a little unconfident at the beginning. But this is a feeling I became to know well throughout my studies and what I learned is that small steps bring you forward to the goal. Therefore we began reading ourselves into the implementation of Ethereum and NEO and soon started to see results. I did always consider this an awesome topic to be working on even thought it was not uncomplicated. At this point, I think we achieved the goals set for this work and this makes me happy and probably also a little bit proud. 

As my partner and I did not know us well at the start of the work there has been one or the other heavily fought discussion about how things should be done. I don’t consider this to be a bad thing even normal, because each conflict can teach you something or it can create synergy if one listens. The most important take away for myself is that in order to resolve discussions before they happen I should formulate the others opinion, thoroughly listen and understand, before I formulate mine. This seemed to result mostly in synergy whereas the other way around mostly in compromise. After all I am grateful for my partners work, help and the things we accomplished together. I also think there is great value in doing the thesis in a team. 

Another important lesson I learned or more an idea I heard about that was confirmed is that humans organize their mind by speaking or writing. Every time when I read the texts or the code I’ve written it became a little clearer and I could formulate it more concise. The interesting thing I’ve noticed is that refactoring and TDD is a tool which helps doing exactly that. Throughout the work I became to see the value of writing tests and refactoring ever more clearly and it is certainly a tool I do wish to constantly use when working at a company.
